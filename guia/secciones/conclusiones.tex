\section{Conclusiones}


Las redes de c�mputo de las organizaciones, se vuelven cada vez m�s complejas y la exigencia de la operaci�n es cada vez mas demandante. Las redes, y cada vez mas, las redes inal�mbrias, soportan aplicaciones y servicios estrat�gicos de las organizaciones o empresas. Por lo cual el an�lisis y monitoreo de redes wi-fi se ha convertido en una labor cada vez mas importante para evitar problemas.\\

La instalaci�n de una red wi-fi no debe concluir al conectar el router, hoy en d�a es una tecnolog�a muy popular y el mantenimiento de la red resulta casi imprescindible, para evitar invasiones de privacidad, interferencias con otros puntos de acceso e incluso con otros electrodom�sticos.\\

Este escaner proporciona informaci�n �til sobre las redes que se encuentran en el entorno, tal como la potencia de la se�al, el canal en que transmite y si la red esta encriptada y con que protocolo.\\

A nivel personal, decir que he disfrutado en la realizaci�n del proyecto y que los conocimientos adquiridos en programaci�n de interfaces gr�ficas abren un abanico de posibilidades para el desarrollo de aplicaciones, un campo que no ha sido estudiado en la carrera y que es casi obligatorio en el �mbito empresarial.\\

El resultado de este proyecto ha resultado muy interesante. Al crearlo desde el principio se ha tenido que realizar una planificaci�n sobre los pasos a seguir y se han tenido que tomar decisiones que afectaban a la realizaci�n del mismo, como la elecci�n del lenguaje de programaci�n o las librer�as gr�ficas a utilizar, pero todo ello sin perder de vista la premisa principal, debe ser extendible a otras plataformas.\\

A pesar de los problemas que han aparecido durante el desarrollo del proyecto, problemas con las librer�as, con la adquisici�n de datos, etc\ldots resulta gratificante ver que el proyecto es funcional y que se han logrado los objetivos establecidos.\\



%COMENTAR EXPERIENCIAS SOBRE EL PROYECTO Y SU ADECUACION AL MISMO EN LA CARRERA \\
%


\section{Posibles extensiones}
Este proyecto podr�a aumentar su funcionalidad:
\begin{itemize}
\item A�adiendo un hist�rico de intensidad de se�al en los canales.
\item Permitiendo la selecci�n de una red en el TreeView para su an�lisis individual.
\item Almacenando los valores tomados en un an�lisis para un an�lisis posterior o ampliaci�n de los datos tomados.
\end{itemize}

