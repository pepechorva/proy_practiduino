Para compilar �ste proyecto se necesita satisfacer una serie de dependencias del c�digo. �stas se obtienen mediante el comando LDD, as� pues, obtenemos esta lista de dependencias del proyecto;
\begin{enumerate}
\item 	linux-vdso.so.1 =>  (0x00007fff3f7d8000)
\item 	libQtGui.so.4 => /usr/lib/libQtGui.so.4 (0x00007f03613ac000)
\item 	libQtCore.so.4 => /usr/lib/libQtCore.so.4 (0x00007f0360f1e000)
\item 	libpthread.so.0 => /lib/libpthread.so.0 (0x00007f0360d01000)
\item 	libstdc++.so.6 => /usr/lib/libstdc++.so.6 (0x00007f03609f7000)
\item 	libgcc_s.so.1 => /usr/lib/libgcc_s.so.1 (0x00007f03607e1000)
\item 	libc.so.6 => /lib/libc.so.6 (0x00007f0360480000)
\item 	libglib-2.0.so.0 => /usr/lib/libglib-2.0.so.0 (0x00007f0360199000)
\item 	libpng14.so.14 => /usr/lib/libpng14.so.14 (0x00007f035ff70000)
\item 	libz.so.1 => /usr/lib/libz.so.1 (0x00007f035fd58000)
\item 	libfreetype.so.6 => /usr/lib/libfreetype.so.6 (0x00007f035fabf000)
\item 	libgobject-2.0.so.0 => /usr/lib/libgobject-2.0.so.0 (0x00007f035f870000)
\item 	libSM.so.6 => /usr/lib/libSM.so.6 (0x00007f035f669000)
\item 	libICE.so.6 => /usr/lib/libICE.so.6 (0x00007f035f44e000)
\item 	libXrender.so.1 => /usr/lib/libXrender.so.1 (0x00007f035f244000)
\item 	libfontconfig.so.1 => /usr/lib/libfontconfig.so.1 (0x00007f035f010000)
\item 	libXext.so.6 => /usr/lib/libXext.so.6 (0x00007f035edfe000)
\item 	libX11.so.6 => /usr/lib/libX11.so.6 (0x00007f035eac0000)
\item 	libm.so.6 => /lib/libm.so.6 (0x00007f035e83d000)
\item 	libdl.so.2 => /lib/libdl.so.2 (0x00007f035e639000)
\item 	libgthread-2.0.so.0 => /usr/lib/libgthread-2.0.so.0 (0x00007f035e435000)
\item 	librt.so.1 => /lib/librt.so.1 (0x00007f035e22d000)
\item 	/lib/ld-linux-x86-64.so.2 (0x00007f036203e000)
\item 	libpcre.so.0 => /lib/libpcre.so.0 (0x00007f035dff2000)
\item 	libuuid.so.1 => /lib/libuuid.so.1 (0x00007f035ddee000)
\item 	libexpat.so.1 => /usr/lib/libexpat.so.1 (0x00007f035dbc5000)
\item 	libxcb.so.1 => /usr/lib/libxcb.so.1 (0x00007f035d9aa000)
\item 	libXau.so.6 => /usr/lib/libXau.so.6 (0x00007f035d7a8000)
\item 	libXdmcp.so.6 => /usr/lib/libXdmcp.so.6 (0x00007f035d5a3000)
\end{enumerate}

Una vez resueltas estas dependencias, dentro de la carpeta del proyecto CAPTURA DE PANTALLA existe un script de compilaci�n e instalaci�n en el sistema, simplemente hay que escribir el comando;
El programa se instala en entorno Gnome en Aplicaciones; internet
CAPTURA DE PANTALLA
