Para compilar �ste proyecto se necesita satisfacer una serie de dependencias del c�digo. �stas se obtienen mediante el comando \texttt{ldd}, as� pues, obtenemos esta lista de dependencias del proyecto;
\begin{verbatim}
linux-vdso.so.1 =>  (0x00007fff3f7d8000)
libQtGui.so.4 => /usr/lib/libQtGui.so.4 (0x00007f03613ac000)
libQtCore.so.4 => /usr/lib/libQtCore.so.4 (0x00007f0360f1e000)
libpthread.so.0 => /lib/libpthread.so.0 (0x00007f0360d01000)
libstdc++.so.6 => /usr/lib/libstdc++.so.6 (0x00007f03609f7000)
libgcc_s.so.1 => /usr/lib/libgcc_s.so.1 (0x00007f03607e1000)
libc.so.6 => /lib/libc.so.6 (0x00007f0360480000)
libglib-2.0.so.0 => /usr/lib/libglib-2.0.so.0 (0x00007f0360199000)
libpng14.so.14 => /usr/lib/libpng14.so.14 (0x00007f035ff70000)
libz.so.1 => /usr/lib/libz.so.1 (0x00007f035fd58000)
libfreetype.so.6 => /usr/lib/libfreetype.so.6 (0x00007f035fabf000)
libgobject-2.0.so.0 => /usr/lib/libgobject-2.0.so.0 (0x00007f035f870000)
libSM.so.6 => /usr/lib/libSM.so.6 (0x00007f035f669000)
libICE.so.6 => /usr/lib/libICE.so.6 (0x00007f035f44e000)
libXrender.so.1 => /usr/lib/libXrender.so.1 (0x00007f035f244000)
libfontconfig.so.1 => /usr/lib/libfontconfig.so.1 (0x00007f035f010000)
libXext.so.6 => /usr/lib/libXext.so.6 (0x00007f035edfe000)
libX11.so.6 => /usr/lib/libX11.so.6 (0x00007f035eac0000)
libm.so.6 => /lib/libm.so.6 (0x00007f035e83d000)
libdl.so.2 => /lib/libdl.so.2 (0x00007f035e639000)
libgthread-2.0.so.0 => /usr/lib/libgthread-2.0.so.0 (0x00007f035e435000)
librt.so.1 => /lib/librt.so.1 (0x00007f035e22d000)
/lib/ld-linux-x86-64.so.2 (0x00007f036203e000)
libpcre.so.0 => /lib/libpcre.so.0 (0x00007f035dff2000)
libuuid.so.1 => /lib/libuuid.so.1 (0x00007f035ddee000)
libexpat.so.1 => /usr/lib/libexpat.so.1 (0x00007f035dbc5000)
libxcb.so.1 => /usr/lib/libxcb.so.1 (0x00007f035d9aa000)
libXau.so.6 => /usr/lib/libXau.so.6 (0x00007f035d7a8000)
libXdmcp.so.6 => /usr/lib/libXdmcp.so.6 (0x00007f035d5a3000)
\end{verbatim}

Una vez resueltas estas dependencias, dentro de la carpeta del proyecto
\figura{0.4}{imgs/capturas/proyecto.png}{}{carpeta}{h}
existe un script de compilaci�n e instalaci�n en el sistema, simplemente hay que escribir el comando;\\

\texttt{./install.sh}\\


El programa se instala en entorno Gnome en Aplicaciones; internet\ver{gnomenu}
\figura{0.8}{imgs/capturas/gnomenu.jpg}{Ubicaci�n del programa}{gnomenu}{h}
